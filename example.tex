\documentclass[10pt]{article}

\usepackage{sparql}
\usepackage{hyperref}
\usepackage{paralist}

\newenvironment{innerlist}[1][\enskip\textbullet]%
        {\begin{compactitem}[#1]}{\end{compactitem}}

\begin{document}
    
Rendering example for sparql{\TeX}. Not super fancy, but you got the idea.
\newline

\textbf{Some events I co-chaired}
%% Running a query on a RDFa enabled file (wordpress site)
\begin{sparql}
data = "http://apassant.net/committees/"
query = """
SELECT ?rolelabel ?eventlabel ?eventhomepage ?toplabel ?tophomepage
WHERE {
<http://apassant.net/alex> swc:holdsRole [
    a swc:Chair ;
    rdfs:label ?rolelabel ;
    swc:isRoleAt [
        a swc:WorkshopEvent ;
        rdfs:label ?eventlabel ;    
        foaf:homepage ?eventhomepage ;
        swc:isSubEventOf [
            rdfs:label ?toplabel ;  
            foaf:homepage ?tophomepage ;
        ]
    ]
]
}"""
tlist = "innerlist"
template = "%(rolelabel)s at \href{%(eventhomepage)s}{%(eventlabel)s} co-located with \href{%(tophomepage)s}{%(toplabel)s}"
\end{sparql}

\vspace{5 mm}
    
\textbf{Me at \href{http://seevl.net}{seevl.net}}
%% Running a query on a LinkedIn profile (uF to RDF using any23)
\begin{innerlist}
\begin{sparql}
data = "http://linkedin.com/in/apassant"
query = """
SELECT DISTINCT ?description
WHERE {
[] ical:description ?description ;
	ical:dtstart '2011-02-01' .
}"""
template = "%(description)s"
\end{sparql}
\end{innerlist}

\end{document}

